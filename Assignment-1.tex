\documentclass[a4paper,12pt]{amsart}
%\usepackage{amssymb}
%\usepackage{amsmath}
\usepackage{mathrsfs}
%\usepackage{amsthm}
\usepackage{cleveref}
\usepackage[doi=false,isbn=false,url=false,natbib=true,sorting=nyt,firstinits=true,style=ieee]{biblatex}
%\addbibresource{RGbib.bib}
\usepackage{showkeys}
\usepackage{esint}
\usepackage{graphicx}
\usepackage{copyrightbox}
\usepackage{color}
\usepackage{dsfont}
\usepackage{tabularx}
%\usepackage{mdwlist}
\usepackage{enumitem}\setlist{leftmargin=*}
\usepackage{etaremune}%\setlist{leftmargin=*}
\usepackage{comment}
\usepackage[super]{nth}
%\usepackage[all=normal,paragraphs,floats,wordspacing]{savetrees}
%\pagestyle{empty}
%\linespread{1.1}
\usepackage[centering]{geometry}
\usepackage{url}
\usepackage[colorlinks=true,urlcolor=blue]{hyperref}
%\usepackage{fancyhdr}
%\lhead{}
%\rhead{}
%\pagestyle{fancy}
%\usepackage{todonotes}
\theoremstyle{definition} \newtheorem{lemma}{Lemma}
\theoremstyle{definition} \newtheorem{definition}[lemma]{Definition}
\theoremstyle{definition} \newtheorem{theorem}[lemma]{Theorem}
\theoremstyle{definition} \newtheorem{question}{Question}
\theoremstyle{definition} \newtheorem{example}[lemma]{Example}
\theoremstyle{remark} \newtheorem{remark}[lemma]{Remark}
\theoremstyle{definition} \newtheorem{corollary}[lemma]{Corollary}
\theoremstyle{definition} \newtheorem{proposition}[lemma]{Proposition}
\theoremstyle{remark} \newtheorem*{notation}{Notation}
\theoremstyle{remark} \newtheorem*{ackn}{Acknowledgements}
\newcommand{\R}{\mathbb{R}}
\newcommand{\N}{\mathbb{N}}
\newcommand{\spt}{\mathrm{spt}}
\newcommand{\df}{\mathrel{\mathop:}=}

\renewcommand{\descriptionlabel}[1]{%
  \hspace\labelsep \upshape\bfseries #1.%
}

\author{}
%	\address{Mathematics Institute, Zeeman Building, University of Warwick, CV4 7AL, UK.}
%	\email{R.Gratwick@warwick.ac.uk}
%\date{\today}
%The research leading to these results has received funding from the European Research Council / ERC Grant Agreement n.291497.\\ %% For work from Warwick ERC grant
%	File: \jobname}

\title{APPS Assignment 1}

\thanks{RTG, Sep 2016}

\begin{document}

\maketitle
\begin{center}
{\it
Hand in solutions at the lecture at 10.00 on Thursday~\nth{29} September.  As with all mathematics, answers to all questions should include full justification.
}
\end{center}
\begin{description}[itemsep=1.0pt]
\item[Exercise~1.2 (edited)] Let $B, C, D$, and $E$ be the following sets:
\begin{align*}
B &=\left\{x \mid x\ \text{is a real number},\ x^2<4\right\},\\
C & =\left\{x \mid x\ \text{is a real number},\ 0\leq x<2\right\},\\
D & =\left\{x \mid x\in\mathbb{Z},\ x^2<1\right\},\\
E & =\left\{1\right\}.
\end{align*}
\begin{enumerate}[label=(\alph*)]
\item Identify a pair of these sets that has the property that neither is
contained in the other.  Is this the only possible answer?

\item You are given that $X$ is one of the sets $B, C, D,$ or $E$,
but you do not know which one. You are also given that $E\subseteq
X$ and $X\subseteq B$. What are the possibilities for the identity of $X$?
\end{enumerate}
\hfill
[{\bf 4 marks}]
\item[Exercise~1.6 (edited)] Write down careful proofs of the following
statements.
\begin{enumerate}[label=(\alph*)]
\item $\sqrt{6}-\sqrt{2}>1$.%
\item If $n$ is an integer such that $n^2$ is odd, then $n$ is odd.%
\item If $n=m^3-m$ for some \emph{odd} integer $m$, then $n$ is a multiple of
$8$.
\end{enumerate}
\hfill
[{\bf 6 marks}]
\item[Exercise~1.10 (edited)] 
Prove by contradiction that a real number that is less than or equal to every positive real number cannot be positive.
\hfill
[{\bf 4 marks}]
\item[Exercise~2.4]\hfill
\begin{enumerate}[label=(\alph*)]
\item Let $a,b$ be rationals and $x$ irrational. Show that if
$\frac{x+a}{x+b}$ is rational, then $a=b$.
\item Let $x,y$ be rationals such that
$\frac{x^2+x+\sqrt{2}}{y^2+y+\sqrt{2}}$ is also rational. Prove
that either $x=y$ or $x+y=-1$.
\end{enumerate}
\hfill
[{\bf 6 marks}]
\item[Exercise~2.2 (edited) {[Non-assessed]}]
Which of the following numbers are rational and
which are irrational?  (You may assume that $\sqrt{3}$ is irrational.)
\begin{enumerate}[label=(\alph*)]
\item $\sqrt{2}+\sqrt{3/2}$.%
\item $1+\sqrt{2}+\sqrt{3/2}$.%
\item $2\sqrt{18}-3\sqrt{8}+\sqrt{4}$.%
%\item $\sqrt{2}+\sqrt{3}+\sqrt{5}$ (\emph{Hint:} For this part you can use the fact that if $n$ is a positive integer that is not a square, then $\sqrt{n}$ is irrational. This follows from Proposition~11.3.%
\item $\sqrt{2}+\sqrt{3}-\sqrt{5+2\sqrt{6}}$.%
\end{enumerate}
\end{description}
\end{document}
